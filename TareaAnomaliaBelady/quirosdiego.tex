\documentclass[10pt, article, natbib]{IEEEtran}
\IEEEoverridecommandlockouts
\usepackage[utf8]{inputenc}
\usepackage[spanish]{babel}
\usepackage[T1]{fontenc}
\usepackage{amsmath}
\usepackage{fancyhdr}
\usepackage{mathtools}
\usepackage{tikz}

\pagestyle{fancy}
\fancyhf{Universidad Nacional de Costa Rica}
\rhead{\thepage}
\lhead{Tarea: Anomalía de Belady}
\rfoot{EIF-212 Sistemas Operativos}
\lfoot{I-2023}

\def\changemargin#1#2{\list{}{\rightmargin#2\leftmargin#1}\item[]}
\let\endchangemargin=\endlist

\DeclarePairedDelimiter\ceil{\lceil}{\rceil}
\DeclarePairedDelimiter\floor{\lfloor}{\rfloor}
\makeatletter
\renewcommand*\l@section{\@dottedtocline{1}{1.5em}{2.3em}}
\makeatother

\onecolumn
\begin{document}

Estudiante: Diego Quirós Artiñano

\textbf{Tarea: Demostrar que la Anomalía de Belady ocurre solo con FIFO.} \\

La Amonalía de Belady ocurre cuando con mayor cantidad de marcos de paginación ocurren más page faults. Esto sabemos que ocurre en el algoritmo de paginación FIFO (first in first out).\\

\[ P: \text{Se utiliza el algoritmo FIFO}\]
\[Q: \text{Ocurre la anomalía de Belady} \]
Sabiendo que la Anomalía de Belady ocurre cuando se utiliza FIFO, entonces.
\[P \implies Q\]

Además sabemos que si esto ocurre entonces
\[\neg Q \implies \neg P\]

Lo que significa que si no ocurre la anomalía de Belady entonces no se utilizó FIFO


\end{document}